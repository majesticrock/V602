\section{Diskussion}
\label{sec:Diskussion}
Alle aus den Grafiken abzulesenden Winkel haben eine Abweichung von $0.35°$, wie die Überprüfung des Bragg-Winkels gezeigt hat. Hinzu kommt, dass
das meschliche Auge nur mit einer gewissen Genauigkeit ablesen kann, womit das Ergebnis verfälscht werden kann. Die Vergleichs- und Literaturwerte, beziehunsgweise
die Abweichungen von diesen sind bereits in den vorherigen Kapiteln bestimmt worden. Im Allgemeinen sind diese als eher gering einzuschätzen, 
was auf eine zumindest insgesamt gute Messung schließen lässt. Auffällig ist nur die Abweichung bei Zirkonium; die Abweichung der bestimmten
Abschirmkonstante vom Literaturwert um $46.29 \%$ fällt gegenüber den Abweichungen der Messung der anderen Elemente deulich auf, was aber 
auf Grund des falsch eingestellten Messbereiches zu Erwarten ist. Daher sind die Werte von Zirkonium bei der Bestimmung der Rydbergenergie
auch nicht weiter beachtet worden. Da die Rydbergenergie hier nur eine Abweichung von $9.06 \%$ aufweist, zeigt allerdings, dass die Messung 
der anderen Materialien in Ordnung sind. Des Weiteren zeigt die Abweichung der maximalen, gemessenen Energie vom Sollwert um $7.31 \%$, dass nicht 
die gesamte Energie in Röntgenstrahlung umgewandelt wird und das Ergebnis dadurch ebenfalls verfälscht wird.Zur Verringerung der Fehler wären Messungen in kleinschrittigeren Winkelabständen, mit jeweils höherer Messzeit, sowie eine digitale 
Auswertung, um eventuelle Ablesefehler zu vermeiden, notwendig.