\section{Diskussion}
\label{sec:Diskussion}
Für alle aus den Grafiken abgelesenen Winkel gilt eine Ungenauigkeit beim Ablesen dieser. Das menschliche Auge ist nur in der Lage mit einer
gewissen Genauigkeit abzulesen, was durch die teilweise ungünstige Skalierung der Diagramme zusätzlich erschwert wird.
Relative Abweichungen von Vergleichs- beziehungsweise Literaturwerten sind bereits im vorherigen Kapitel bestimmt worden. Diese sind im Allgemeinen
gering, was auf eine insgesamt gute Messung schließen lässt. Auffällig ist hier die Messung der K-Kante von Zirkonium. Eine relative Abweichung
von $44.47 \%$ bei den Abschirmkonstanten, lässt darauf schließen, dass diese Messung als schlechter einzustufen ist. Hier ist die 
Skalierung vermutlich schlecht angepasst worden, womit die tatsächliche K-Kante wahrscheinlich außerhalb des eingestellten Messbereiches
liegt. Dass die spätere Bestimmung der Rydbergenergie nur eine relative Abweichung von $5.44 \%$ aufweist, zeigt, dass die Ungenauigkeiten bei der Messung der Abschirmkonstante von 
Zirkonium hier keine gravierenden Folgen hat, so dass das Ergebnis dennoch gut ist. Auch die Abweichung der maximalen, gemessen Energie vom Sollwert
um $7.35 \%$ zeigt, dass nicht die gesamte Energie in Röntgenstrahlung umgewandelt wird und das Ergebnis dadurch auch verfälscht. 
Zur Verringerung der Fehler wären Messungen in kleinschrittigeren Winkelabständen, mit jeweils höherer Messzeit, sowie eine digitale 
Auswertung, um eventuelle Ablesefehler zu vermeiden, notwendig. 