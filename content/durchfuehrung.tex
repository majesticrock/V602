\section{Durchführung}
\label{sec:Durchführung}

Der Aufbau des Versuchs besteht aus einer Röntgenröhre mit einer Glühkathode und einer Kupferanode.
Die Elektronen aus der Glühkathode werden mit $U_\text{B} = 35$ kV beschleunigt. Der Röntgenstrahl ist so gerichtet, dass er auf einen LiF-Kristall ($d = 201,4$ pm) trifft und an diesem reflektiert wird.

Das Kristallwinkel kann eingestellt werden.
Ebenfalls befindet sich in dem Gerät ein Geiger-Müller-Zählrohr, dass sich winkelabhängig verschieben lässt und so die Strahlungsintensität an einem bestimmten Winkel messen kann.

Die Apperatur selbst wird mittels eines Computerprogramms bedient.
An diesem wird eingestellt in welchem Winkel der Kristall stehen soll, in welchen Zeitintervall das Geiger-Müller-Zählrohr bewegt werden soll (=Integrationszeit) und bei welchem Winkel dies beginnen und Enden soll, sowie die Größe der Schritte.

