\section{Durchführung}
\label{sec:Durchführung}

Der Aufbau des Versuchs besteht aus einer Röntgenröhre mit einer Glühkathode und einer Kupferanode.
Die Elektronen aus der Glühkathode werden mit $U_\text{B} = 35$ kV beschleunigt. Der Röntgenstrahl ist so gerichtet, dass er auf einen LiF-Kristall ($d = 201,4$ pm) trifft und an diesem reflektiert wird.

Das Kristallwinkel kann eingestellt werden.
Ebenfalls befindet sich in dem Gerät ein Geiger-Müller-Zählrohr, dass sich winkelabhängig verschieben lässt und so die Strahlungsintensität an einem bestimmten Winkel messen kann.

Die Apperatur selbst wird mittels eines Computerprogramms bedient.
An diesem wird eingestellt in welchem Winkel der Kristall stehen soll, in welchem Zeitintervall das Geiger-Müller-Zählrohr bewegt werden soll (=Integrationszeit) und bei welchem Winkel dies beginnen und enden soll, sowie die Größe der Schritte.
Vor Versuchsbeginn wird überprüft, dass die Schlitzblende auf dem Geiger-Müller-Zählrohr senkrecht zur Drehrichtung ausgerichtet ist.

Zunächst wird die Bragg-Bedingung überprüft. Dazu wird das Kristallwinkel auf $\theta = 14°$ gestellt. Das Geiger-Müller-Zählrohr wird in dem Winkelbereich von $\alpha_\text{GM} = 26°$ bis $\alpha_\text{GM} = 30°$ mit einer Schrittgröße von $\Delta \alpha = 0,1°$ gedreht.
Die Messart wird auf "Spektren" gestellt.

Als nächstes wird das Emissionsspektrum der Cu-Röntgenröhre gemessen.
Dazu wird in dem Programm der "2:1 Koppelmodus" ausgewählt. Es wird das Röntgenspektrum im Winkelbereich von $\alpha_\text{GM} = 4°$ bis $\alpha_\text{GM} = 26°$ gemessen.
Dabei werden jeweils $0,2°$ Schritte gemacht und die Integrationszeit betrage $\Delta t = 5$ s.

Nun wird Absorption von verschiedenen Elementen untersucht.

Zunächst werden vier verschiedene Elemente mit einer Kernladungszahl im Bereich von $30 \leq Z \leq 50$ untersucht.
Dazu wird je eines vor das Geiger-Müller-Zählrohr geschraubt. Der Winkelbereich in dem gemessen wird, wird bei den jeweiligen Elemente so gewählt, dass die K-Linien in diesem gut zu sehen sind.
Die Schrittgröße betrage $\Delta \alpha = 0,1°$ und die Integrationszeit sei $\Delta t = 20$ s.

Zuletzt wird die Absorptionsmessung mit einem Absorber mit einer Kernladungszahl $70 \leq Z$. Die Integrationszeit und Schrittgröße bleibe unverändert.
Der Messbereich werde diesmal so gewählt, dass die L-Linien gut zu sehen sind.
