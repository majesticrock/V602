\section{Auswertung}
\label{sec:Auswertung}

\subsection{Überprüfung der Bragg-Bedingung}
    Zur Überprüfung der Bragg-Bedingung wird die Abbildung 1 aus dem Anhang verwendet. 
    Hier der Winkel unter dem das Maximum zu sehen ist aufgenommen. Dieser liegt bei 
    \\ \\
    \centerline{$\theta = 13.65°$}
    \\ \\
    und weicht damit um $2.5\%$ vom eingestelltem Winkel $\theta = 14°$ ab.



\subsection{Das Emmissionsspektrum von Kupfer}
    Für diesen Auswertungsteil wird das Emmissionsspektrum von Kupfer, dargestellt in Abbildung 2
    aus dem Anhang, verwendet. 
    \subsubsection{Maximalenergie des Bremsspektrums}
        Zur Bestimmung der maximalen Energie des Bremsspektrums von Kupfer wird der kleinste Winkel aus der Grafik abgelesen, bei dem sich 
        das Spektrum von Null unterscheidet. Dieser liegt bei 
        \\ \\
        \centerline{$\theta = 4.70° $.}
        \\ \\
        Mittels Gleichung GLEICHUNGENERGIEWINKEL ergibt sich damit eine maximale Energie von
        \\ \\
        \centerline{$E_{\symup{max}} = 37.57$keV}
        \\ \\ 
        und damit nach Gleichung GLEICHUNGWINKELWELLENLÄNGE eine minimale Wellenlänge von
        \\ \\
        \centerline{$\lambda_{\symup{min}} = 3.3 \cdot 10^{-11}$m.}
        \\ \\
        Der Theoriewert liegt nach Gleichung GLEICHUNGREINEENERGIE bei 
        \\ \\
        \centerline{$E_{\symup{theorie}} = 35$keV}
        \\ \\
        und damit um $7.35 \%$ unterhalb des gemessenen Wertes.

    \subsubsection{Auflösungsvermögen der Apparatur} 
        Um das Auflösungsvermögen der Apparatur zu bestimmen werden die Winkel abgelesen an denen die $K_{\symup{\alpha}}$ und die $K_{\symup{\beta}}$
        Line auf die Hälfte ihres Maximums abgesunken sind.
        Diese liegen bei:
        \\ \\
        \centerline{$\theta_{\symup{\alpha}} = 22.50°$}
        \centerline{$\theta_{\symup{\beta}} = 20.29°$.}
        \\ \\
        Die zugehörigen Energien berechnen sich nach GLEICHUNGENERGIEWINKEL zu
        \\ \\
        \centerline{$E_{\symup{\alpha}} = 8.04$ keV}
        \\ \\
        und 
        \centerline{$E_{\symup{\beta}} = 8.88$ keV,}
        \\ \\
        womit sich eine Energiedifferenz von 
        \\ \\
        \centerline{$\Delta E = E_{\symup{\beta}} - E_{\symup{\alpha}} = 0.83$eV,}
        \\ \\
        was auch die Energieauflösung ist.
        Dadurch ergibt sich eine Wellenlängenauflösung von 
        \\ \\
        \centerline{$\lambda_{\symup{Aufl}} = 1.49 \cdot 10^{-6}$.}
        \\ \\

        STATISTISCHER FEHLER
    \subsubsection{Bestimmung der Abschirmkonstante}
        Es werden die Winkel aufgenommen unter denen die $K_{\symup{\alpha}}$ beziehungsweise die $K_{\symup{\beta}}$ auftreten.
        Sie liegen bei
        \\ \\
        \centerline{$\theta_{\symup{K_{\symup{\alpha}}}} = 22.21°$}
        \\ \\
        und 
        \\ \\
        \centerline{$\theta_{\symup{K_{\symup{\beta}}}} = 20.00°$.}
        \\ \\
        Durch Gleichung GLEICHUNGENERGIEWINKEL lassen sich die zugehörigen Engergien, 
        \\ \\
        \centerline{$E_{\symup{\alpha}} = 8.14 $keV}
        \centerline{$E_{\symup{\beta}} = 9.00 $keV.}
        \\ \\
        berechnen.
        Die Abschirmkonstanten berechnen sich dann zu:
        \\ \\
        \centerline{$\sigma_1 = z - \sqrt{\frac{E_{\symup{\beta}}}{R_{\inf}} } = 3.46 $}
        \centerline{$\sigma_2 = z - 2 \sqrt{\frac{R_{\inf}(z - \sigma_1)^2 - E_{\symup{\alpha}}}{R_{\inf}}} = 13.24 $,}
        \\ \\
        wobei $z = 29$ die Ordungszahl von Kupfer und $R_{\infty} = 13.6 \symup{eV}$ die Rydbergenergie ist.
\subsection{Absorbstionsspektren verschiedener Elemente}    
    In diesem Versuchsteil sind Absorbstionsspektren verschiedener Elemente aufgenommen worden. Aus diesen werden die Energien der 
    K-Kanten entnommen und daraus die zugehörigen Abschirmzahlen bestimmt. Diese werden mit den zuvor berechneten Werten verglichen.
        \begin{table}[!htp]
\centering
\caption{K-Kanten und zugehörige Abschirmkonstante, sowie Literaturwerte zum Vergleich.}
\label{tab:messung}
\begin{tabular}{c c c c c c c}
\toprule
{{Material}} & {{$z$}} & {{$\theta$}} & {{$E_{\symup{K, ber}}$}}  & {{$E_{\symup{K, Lit}}$}}  & {{$\sigma_{\symup{K, ber}}$}} & {{$\sigma_{\symup{K, Lit}}$}} \\
\midrule
Brom &      35  & 13.4 & 13.29 & 13.474 & 3.74  & 3.52 \\
Zink &      30  & 18.9 & 9.51  &  9.659 & 3.56  & 3.35 \\
Strontium & 38  & 11.2 & 15.85 & 16.105 & 4.58  & 3.58 \\
Zirkonium & 40  & 10.8 & 16.43 & 17.998 & 5.23  & 3.62 \\
\bottomrule
\end{tabular}
\end{table}




