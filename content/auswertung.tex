\section{Auswertung}
\label{sec:Auswertung}

\subsection{Überprüfung der Bragg-Bedingung}
    Zur Überprüfung der Bragg-Bedingung wird die Abbildung 1 aus dem Anhang verwendet. 
    Hierzu wird der Winkel unter dem das Maximum zu sehen ist, aufgenommen. Dieser liegt bei 
    \\ \\
    \centerline{$\theta = 13.65°$}
    \\ \\
    und weicht damit um $0.35°$ ($2.5\%$) vom eingestelltem Winkel $\theta = 14°$ ab.
    Daher weisen alle weiteren gemessenen Winkel den gleichen Fehler auf.

\subsection{Das Emmissionsspektrum von Kupfer}
    Für diesen Auswertungsteil wird das Emmissionsspektrum von Kupfer, dargestellt in Abbildung 2
    aus dem Anhang, verwendet. 
    \subsubsection{Maximalenergie des Bremsspektrums}
        Zur Bestimmung der maximalen Energie des Bremsspektrums von Kupfer wird der kleinste Winkel aus der Grafik abgelesen, bei dem sich 
        das Spektrum von Null unterscheidet. Dieser liegt bei 
        \\ \\
        \centerline{$\theta = (4.70  \pm 0.35)° $.}
        \\ \\
        Mittels Gleichung \eqref{eqn:bragg-energie} ergibt sich damit eine maximale Energie von
        \\ \\
        \centerline{$E_{\symup{max}} = (37.76 \pm 2.8)$keV}
        \\ \\ 
        und damit nach Gleichung \eqref{eqn:lambda-min} eine minimale Wellenlänge von
        \\ \\
        \centerline{$\lambda_{\symup{min}} = (3.3 \pm 0.25) \cdot 10^{-11}$m.}
        \\ \\
        Die Fehler der Gesamtenergie beziehungsweise der Wellenlänge ergeben sich mittels Gaußscher Fehlerfortpflanzung zu 
        \begin{equation}
        \label{eqn:fehlerE}
            \Delta E = \frac{h c}{2 d} \cdot \frac{\cos{\theta}}{\sin^2{\theta}} \Delta \theta
        \end{equation}
        und 
        \begin{equation}
        \label{eqn:fehlerL}
            \Delta \lambda = \frac{h c}{E^2} \Delta E,
        \end{equation}
        wobei $h$ das Plancksche Winkungsquantum, $c$ die Lichtgeschwindigkeit, $d$ die Gitterkonstante und die mit $\Delta$ bezeichneten
        Größen die absoluten Fehler dieser sind. 
        Der Theoriewert liegt  bei 
        \\ \\
        \centerline{$E_{\symup{theorie}} = e_0 \cdot U = 35$keV}
        \\ \\
        und damit um $7.31 \%$ unterhalb des gemessenen Wertes, wobei $e_0$ die Elementarladung und $U$ die angelegte Spannung ist.

    \subsubsection{Auflösungsvermögen der Apparatur}
        \label{sec:cu} 
        Um das Auflösungsvermögen der Apparatur zu bestimmen werden die Winkel abgelesen an denen die $K_{\symup{\alpha}}$ und die $K_{\symup{\beta}}$
        Line auf die Hälfte ihres Maximums abgesunken sind.
        Diese liegen bei:
        \\ \\
        \centerline{$\theta_{\symup{\alpha,1}} = (22.06 \pm 0.35)°$}
        \centerline{$\theta_{\symup{\alpha,2}} = (22.64 \pm 0.35)°$}
        \centerline{$\theta_{\symup{\beta,1}} = (20.00 \pm 0.35)°$}
        \centerline{$\theta_{\symup{\beta,2}} = (20.44 \pm 0.35)°$.}
        \\ \\
        Aus den Winkeln ergeben sich mittels Gleichung \eqref{eqn:bragg-energie} und \eqref{eqn:fehlerE} die zugehörigen Energien
        \\ \\
        \centerline{$E_{\symup{\alpha,1}} = (8.20 \pm 0.12)$keV}
        \centerline{$E_{\symup{\alpha,2}} = (8.00 \pm 0.12)$keV}
        \centerline{$E_{\symup{\beta,1}} =  (9.00 \pm 0.15)$keV}
        \centerline{$E_{\symup{\beta,2}} =  (8.81 \pm 0.14)$keV,}
        \\ \\
        woraus sich die Engergiedifferenzen
        \\ \\
        \centerline{$E_{\symup{\alpha}} = (0.2 \pm 0.17)$keV}
        \centerline{$E_{\symup{\beta}} = (0.19 \pm 0.21)$keV}
        \\ \\
        ergeben. Über diese Differenzen wird nun gemittelt um eine Energieauflöung von
        \\ \\
        \centerline{$E = (0.20 \pm 0.14) $keV.}
        \\ \\
    \subsubsection{Bestimmung der Abschirmkonstante}
        Es werden die Winkel aufgenommen unter denen die $K_{\symup{\alpha}}$- beziehungsweise die $K_{\symup{\beta}}$-Linien auftreten.
        Sie liegen bei
        \\ \\
        \centerline{$\theta_{\symup{K_{\symup{\alpha}}}} = (22.21 \pm 0.35)°$}
        \\ \\
        und 
        \\ \\
        \centerline{$\theta_{\symup{K_{\symup{\beta}}}} = (20.00 \pm 0.35)°$.}
        \\ \\
        Durch die Gleichungen \eqref{eqn:bragg-energie} und \eqref{eqn:fehlerE} lassen sich die zugehörigen Engergien, 
        \\ \\
        \centerline{$E_{\symup{\alpha}} = (8.14 \pm 0.12) $keV}
        \centerline{$E_{\symup{\beta}} = (9.00 \pm 0.15) $keV.}
        \\ \\
        berechnen.
        Der Theoriewert der K-Kante liegt bei $K = 8.979 \symup{keV}$ \cite[periodic], über die die erste Abschirmkonstante berechnet werden kann.
        Die Abschirmkonstanten berechnen sich dann zu:
        \\ \\
        \centerline{$\sigma_1 = z - \sqrt{\frac{E_{\symup{K}}}{R_{\infty}} } = 3.49 $,}
        \centerline{$\sigma_2 = z - 2 \sqrt{\frac{R_{\infty}(z - \sigma_1)^2 - E_{\symup{\alpha}}}{R_{\infty}}} = 13.4 \pm 1.1 $.}
        \\ \\
        Die dritte Abschirmkonstante wird über 
        \\ \\
        \centerline{$\sigma_3 = z - 2 \sqrt{\frac{R_{\infty}(z - \sigma_1)^2 - E_{\symup{\beta}}}{R_{\infty}}}$}
        \\ \\
        berechnet, was hier allerdings nicht möglich ist, da der Wurzelterm negativ wird, was auf eine ungenaue Messung von $E_{\symup{\beta}}$ schließen lässt.
        In allen Rechnungen ist $z = 29$ die Ordungszahl von Kupfer und $R_{\infty} = 13.6 \symup{eV}$ die Rydbergenergie.

\subsection{Bestimmung der Abschirmkonstanten verschiedener Elemente} 
\label{sec:bah}
    In diesem Versuchsteil sind Absorbstionsspektren verschiedener Elemente aufgenommen worden. Aus diesen werden die Energien der 
    K-Kanten entnommen und daraus die zugehörigen Abschirmzahlen bestimmt. Diese werden mit Literaturwerten verglichen.
        
        \subsubsection{Zink}
            Das Absorptionsspektrum von Zink ist in Abbildung 3 des Anhanges zu sehen.
            Die K-Kante von Zink ist bei einem Winkel von $(18.9 \pm 0.35)°$ zu sehen. Der darauffolgende
            wesentlich stärkere Anstieg bei ca. $20°$ kann mit Abschnitt 4.2.2 als 
            $\symup{K_{\symup{\beta}}}$-Linie des Kupfers identifiziert werden und wird
            daher hier nicht weiter beachtet. Mittels Gleichung \eqref{eqn:bragg-energie} und \eqref{eqn:fehlerE} lässt sich
            die Absorptionsenergie an der K-Kante zu 
            \\ \\
            \centerline{$E_{\symup{K, Zn}} = (9.500 \pm 0.17)$keV}
            \\ \\
            bestimmen, wodurch eine Abweichung von $1.67 \%$ vom Literaturwert
            \\ \\
            \centerline{$E_{\symup{Lit, Zn}} = 9.659$ keV \cite{periodic}}
            \\ \\
            vorliegt. Über die Gleichung \eqref{eqn:fein-struktur} wird die Abschirmkonstante 
            \\ \\
            \centerline{$\sigma_{\symup{K, Zn}} = (3.76 \pm 0.23)$}
            \\ \\
            berechnet, wobei die Ordnungszahl von Zink $Z = 30$ ist.
            Über die selbe Gleichung wird ein Vergleichswert mit dem Literaturwert der Energie 
            errechnet. Dieser ist
            \\ \\
            \centerline{$\sigma_{\symup{Lit, Zn}} = 3.54$,}
            \\ \\
            womit eine Abweichung von $5.85 \%$ vorliegt.
        \subsubsection{Brom}
            In diesem Teil der Auswertung wird Abbildung 4 aus dem Anhang verwendet.
            Bei Brom ist die K-Kante bei einem Winkel von $(13.4 \pm 0.35)°$ zu sehen. Nach 
            Gleichung \eqref{eqn:bragg-energie} ergibt sich dadurch eine Absorptionsenergie
            von 
            \\ \\
            \centerline{$E_{\symup{K, Br}} = (13.280 \pm 0.34)$keV,}
            \\ \\
            welche somit $1.44 \%$ vom Literaturwert
            \\ \\
            \centerline{$E_{\symup{Lit, Br}} = 13.474$ keV \cite{periodic}}
            \\ \\
            abweicht. Mit Gleichung \eqref{eqn:fein-struktur} errechnet sich die 
            zugehörige Abschirmkonstante zu 
            \\ \\
            \centerline{$\sigma_{\symup{K, Br}} = (4.00 \pm 0.4)$,}
            \\ \\
            welche vom zuvor bestimmten Wert
            \\ \\
            \centerline{$\sigma_{\symup{Lit, Br}} = 3.75$}
            \\ \\
            um $6.25 \%$ abweicht. Die einzusetzende Ordnungszahl von Brom 
            ist $Z = 35$.
        \subsubsection{Strontium}
            Das Absorptionsspektrum von Strontium ist in Abbildung 5 des Anhanges zu sehen.
            Das Maximum wird hier bei einem Winkel von $(11.2 \pm 0.35)°$ erreicht, womit sich nach 
            Gleichung \eqref{eqn:bragg-energie} eine Absorptionsenergie von 
            \\ \\
            \centerline{$E_{\symup{K, Sr}} = (15.8 \pm 0.5)$keV}  
            \\ \\
            ergibt. Verglichen mit dem Literaturwert
            \\ \\
            \centerline{$E_{\symup{Lit, Sr}} = 16.105$ keV \cite{periodic}}
            \\ \\
            ergibt sich eine relative Abweichung von $1.93 \%$. Die Abschirmkonstante
            errechnet sich wieder über Gleichung \eqref{eqn:fein-struktur}, wobei die
            Ordnungszahl hier $Z = 38$ ist, zu
            \\ \\
            \centerline{$\sigma_{\symup{K, Sr}} = (4.1 \pm 0.5)$.} 
            \\ \\
            Diese weicht um $6.34 \%$ vom zuvor bestimmten Wert von 
            \\ \\
            \centerline{$\sigma_{\symup{Lit, Sr}} = 3.84$}
            \\ \\
            ab.

        \subsubsection{Zirkonium}
            Aus Abbildung 6 des Anhangs ist die K-Kante von Zirkonium bei einem 
            Winkel von $(10.8 \pm 0.35)°$ abzulesen. Dieser Wert ist der kleinste aus der Abbildung ablesbare Winkel. Die 
            tatsächliche Kante liegt außerhalb des dargestellten Spektrums, weshalb die hier errechneten Werte nicht den tatsächlichen 
            Werten entsprechen.
            Damit errechnet sich über Gleichung \eqref{eqn:bragg-energie} die Absorptionsenergie
            \\ \\
            \centerline{$E_{\symup{K, Zr}} = (16.400 \pm 0.5)$keV,}  
            \\ \\
            welche eine Abweichung von $8.87 \%$ vom Literaturwert
            \\ \\
            \centerline{$E_{\symup{Lit, Zr}} = 17.998$ keV \cite{periodic}}
            \\ \\
            abweicht. Mittels Gleichung \eqref{eqn:fein-struktur} und der Ordnungszahl $Z = 40$
            ist die Abschirmkonstante
            \\ \\
            \centerline{$\sigma_{\symup{K, Zr}} = (3.5 \pm 0.6)$} 
            \\ \\
            und weicht damit um $46.29 \%$ vom Vergleichswert
            \\ \\
            \centerline{$\sigma_{\symup{Lit, Zr}} = 1.88$}
            \\ \\
            ab.
    \subsection{Bestimmung der Rydbergkonstante}  
        Die Ordnungszahlen der zuvor untersuchten Elemente werden gegen die Wurzel der in \autoref{sec:bah} bestimmten Absorptionsenergien
        aufgetragen. Dies ist in \autoref{fig:ekz} dargestellt. Dabei werden die Werte von Zirkonium ausgelassen, da diese, wie bereits 
        erwähnt nicht den tatsächlichen entsprechen und damit das Ergebnis verfälschen würden.
        Über die Punkte wird mittels Python 3.7.0 eine lineare Regression durchgeführt, sodass sich eine Funktion der Form
        \begin{equation}
        \label{eqn:regress}
            \sqrt{E_{\symup{K}}} = a \cdot Z + b
        \end{equation}
        ergibt. Dadurch ergeben sich folgende Werte:
        \\ \\
        \centerline{$a = (3.53 \pm 0.02) \frac{\sqrt{\symup{eV}}}{\text{Ordnungszahl}} $}
        \\ \\
        \centerline{$b = (-8.44 \pm 0.61 \sqrt{\symup{eV}}$.}
        \\ \\
        Die Rydbergenergie lässt sich aus der Steigung $a$ ermitteln und beträgt
        \\ \\
        \centerline{$R_{\infty, gem} = (12.47 \pm 0.0003)$ eV.}
        \\ \\
        Dieser Wert hat eine Abweichung von $9.06 \%$ von der tatsächlichen Rydbergenergie
        \\ \\
        \centerline{$R_{\infty} = 13.6$ eV.}
        \\ \\
        \begin{figure}
            \centering
            \includegraphics{ekz.pdf}
            \caption{Ordungszahl $Z$ gegen die Wurzeln der Apsorbtionsenergien an den K-Linien mit Ausgleichsgerade.}
            \label{fig:ekz}
        \end{figure}

        \subsection{Bestimmung der Abschirmkonstanten von Gold}
        Das Absorptionsspektrum von Gold ist in Abbildung 7 des Anhanges zu sehen. Daraus sind die Winkel der auftretenden L-Kanten zu entnehmen.
        Diese liegen bei:
        \\ \\
        \centerline{$\theta_{\symup{L,3}} = (15.2 \pm 0.35)°$,}
        \\ \\
        \centerline{$\theta_{\symup{L,2}} = (13.0 \pm 0.35)°$.}
        \\ \\
        Mittels Gleichung \eqref{eqn:bragg-energie} und \eqref{eqn:fehlerE} lassen sich die Energien der beiden Kanten zu
        \\ \\
        \centerline{$E_{\symup{L,3}} = (11.74 \pm 0.26)$ keV,}
        \\ \\
        \centerline{$E_{\symup{L,2}} = (13.70 \pm 0.4)$ keV,}
        \\ \\
        berechnen, woraus sich die Energiedifferenz zu
        \\ \\
        \centerline{$\delta E = E_{\symup{L,2}} - E_{\symup{L,3}} = (1.9 \pm 0.4)$ keV}
        \\ \\
        berechnet. Die Abschirmkonstante berechnet sich dann nach Gleichung \eqref{eqn:abschirm} mit der Ordnungszahl $Z = 79$zu 
        \\ \\
        \centerline{$\sigma_{\symup{L}} = (2 \pm 4)$.}
        \\ \\
        



